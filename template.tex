\documentclass[10pt, letterpaper]{article}

% Packages:
\usepackage[
    ignoreheadfoot, % set margins without considering header and footer
    top=2 cm, % seperation between body and page edge from the top
    bottom=2 cm, % seperation between body and page edge from the bottom
    left=2 cm, % seperation between body and page edge from the left
    right=2 cm, % seperation between body and page edge from the right
    footskip=1.0 cm, % seperation between body and footer
    % showframe % for debugging 
]{geometry} % for adjusting page geometry
\usepackage{titlesec} % for customizing section titles
\usepackage{tabularx} % for making tables with fixed width columns
\usepackage{array} % tabularx requires this
\usepackage[dvipsnames]{xcolor} % for coloring text
\definecolor{primaryColor}{RGB}{0, 79, 144} % define primary color
\usepackage{enumitem} % for customizing lists
% \usepackage{fontawesome5} % for using icons
\usepackage{amsmath} % for math
\usepackage[
    pdftitle={John Doe's CV},
    pdfauthor={John Doe},
    pdfcreator={LaTeX with RenderCV},
    colorlinks=true,
    urlcolor=primaryColor
]{hyperref} % for links, metadata and bookmarks
\usepackage[pscoord]{eso-pic} % for floating text on the page
\usepackage{calc} % for calculating lengths
\usepackage{bookmark} % for bookmarks
\usepackage{lastpage} % for getting the total number of pages
\usepackage{changepage} % for one column entries (adjustwidth environment)
\usepackage{paracol} % for two and three column entries
\usepackage{ifthen} % for conditional statements
\usepackage{needspace} % for avoiding page brake right after the section title
\usepackage{iftex} % check if engine is pdflatex, xetex or luatex

% Ensure that generate pdf is machine readable/ATS parsable:
\ifPDFTeX
    \input{glyphtounicode}
    \pdfgentounicode=1
    % \usepackage[T1]{fontenc} % this breaks sb2nov
    \usepackage[utf8]{inputenc}
    \usepackage{lmodern}
\fi



% Some settings:
\AtBeginEnvironment{adjustwidth}{\partopsep0pt} % remove space before adjustwidth environment
\pagestyle{empty} % no header or footer
\setcounter{secnumdepth}{0} % no section numbering
\setlength{\parindent}{0pt} % no indentation
\setlength{\topskip}{0pt} % no top skip
\setlength{\columnsep}{0cm} % set column seperation
\makeatletter

\titleformat{\section}{\needspace{4\baselineskip}\bfseries\large}{}{0pt}{}[\vspace{1pt}\titlerule]

\titlespacing{\section}{
    % left space:
    -1pt
}{
    % top space:
    0.3 cm
}{
    % bottom space:
    0.2 cm
} % section title spacing

\renewcommand\labelitemi{$\circ$} % custom bullet points
\newenvironment{highlights}{
    \begin{itemize}[
        topsep=0.10 cm,
        parsep=0.10 cm,
        partopsep=0pt,
        itemsep=0pt,
        leftmargin=0.4 cm + 10pt
    ]
}{
    \end{itemize}
} % new environment for highlights

\newenvironment{highlightsforbulletentries}{
    \begin{itemize}[
        topsep=0.10 cm,
        parsep=0.10 cm,
        partopsep=0pt,
        itemsep=0pt,
        leftmargin=10pt
    ]
}{
    \end{itemize}
} % new environment for highlights for bullet entries


\newenvironment{onecolentry}{
    \begin{adjustwidth}{
        0.2 cm + 0.00001 cm
    }{
        0.2 cm + 0.00001 cm
    }
}{
    \end{adjustwidth}
} % new environment for one column entries

\newenvironment{twocolentry}[2][]{
    \onecolentry
    \def\secondColumn{#2}
    \setcolumnwidth{\fill, 4.5 cm}
    \begin{paracol}{2}
}{
    \switchcolumn \raggedleft \secondColumn
    \end{paracol}
    \endonecolentry
} % new environment for two column entries

\newenvironment{header}{
    \setlength{\topsep}{0pt}\par\kern\topsep\centering\linespread{1.5}
}{
    \par\kern\topsep
} % new environment for the header

% save the original href command in a new command:
\let\hrefWithoutArrow\href

% new command for external links:
\renewcommand{\href}[2]{\hrefWithoutArrow{#1}{\ifthenelse{\equal{#2}{}}{ }{#2 }\raisebox{.15ex}{\footnotesize \faExternalLink*}}}


\begin{document}
    \newcommand{\AND}{\unskip
        \cleaders\copy\ANDbox\hskip\wd\ANDbox
        \ignorespaces
    }
    \newsavebox\ANDbox
    \sbox\ANDbox{}

    \begin{header}
        \textbf{\fontsize{24 pt}{24 pt}\selectfont Frank Fu}

        \vspace{0.05 cm}

        \normalsize
        \mbox{{y.f.fu@uq.edu.au}}%
        \kern 0.25 cm%
        \AND%
        \kern 0.25 cm%
        \AND%
        \mbox{{github.com/BSB47}}%
    \end{header}

    \vspace{0.1 cm}


    \section{Summary}
    I solve problems in biochemistry using machine learning and molecular dynamics.
    \section{Education}

        \begin{onecolentry}
          \textbf{Ph.D. Computational Chemistry} -- The University of Queensland \hfill Jul 2023 - Present\\
          \textbf{MSc. Biological Sciences} -- The University of Auckland \hfill Mar 2023\\
          \textbf{BSc. Biological Sciences} -- The University of Auckland \hfill Mar 2021
        \end{onecolentry}
    
    \section{Experience}

    \begin{onecolentry}
      \textbullet{} \textbf{Research Intern} QDX, Melbourne \hfill Feb - Jun 2025
        \begin{highlightsforbulletentries}
            \begin{highlightsforbulletentries}
              \item Implemented novel machine learning framework for automatic recognition and deployment of collective variables in proprietary molecular dynamics software
              \item 4x speedup compared to tradition enhanced sampling workflows, in addition to requiring minimal prior knowledge of the system under study
            \end{highlightsforbulletentries}

        \end{highlightsforbulletentries}
    \end{onecolentry}
    \begin{onecolentry}
      \textbullet{} \textbf{Lab Demonstrator} The University of Auckland \hfill Mar 2021 - May 2023
      \begin{highlightsforbulletentries}
          \begin{highlightsforbulletentries}
          \item \textit{BIOSCI 107: Biology for Biomedical Science} -- microbiology lab techniques such as culturing, microscopy, and gel electrophoresis
          \item \textit{BIOSCI 220: Quantitative Biology} -- statistical modelling/analysis using the R programming language
          \item \textit{BIOSCI 355: Genomics} -- bioinformatics techniques such as sequence alignment, structure prediction, and phylogenetics
          \end{highlightsforbulletentries}
    \end{highlightsforbulletentries}
    \end{onecolentry}
    \begin{onecolentry}
      \textbullet{} \textbf{Research Assistant} The University of Auckland \hfill Dec 2021 - Mar 2022
      \begin{highlightsforbulletentries}
          \begin{highlightsforbulletentries}
          \item Performed experimental evolution of genetically modified \textit{S. cerevisiae} to observe evolutionary dynamics between competing populations
          \item Experiment results were used to contribute to a successful grant application
          \end{highlightsforbulletentries}
      \end{highlightsforbulletentries}

    
    \section{Skills}
    \textbf{Programming} C/C++, Python, R\\
    \textbf{Machine learning} Supervised learning with PyTorch/libtorch, scikit-learn, DGL\\
    \textbf{Molecular dynamics} \textit{classical}: GROMACS, PLUMED, OpenMM \textbar{} \textit{quantum}: xTB

    
    \section{Publications}
    \begin{onecolentry}
    \textbullet{} Fu, Y., \& Takeuchi, N. (2024). Evolution of the division of labour between templates and catalysts in spatial replicator models. \textit{Journal of Evolutionary Biology, 37(10}}, 1158-1169.\\
    \textbullet{} Fu, Y., Stroet, M., \& O'Mara, M. L. (2025). A graph-based machine learning framework to assign empirical interaction parameters for novel molecules. In review.
    \end{onecolentry}

    \section{Awards}
    The University of Queensland Industry Placement Scholarship \hfill 2025\\
    The University of Queensland Earmarked PhD Scholarship \hfill 2023\\
    Best Poster, The Association of Molecular Modellers of Australasia Conference \hfill 2023

    \section{Conference presentations}
    Selected talk at The Association of Molecular Modellers of Australasia Conference \hfill 2025\\
    Selected talk at QUIBIC Symposium \hfill 2025\\
    Selected talk at the Annual Conference of the Australian Society for Biophysics \hfill 2023
\end{document}
